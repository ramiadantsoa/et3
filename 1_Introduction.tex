\section{Introduction}
\label{sec:Introduction}

Biodiversity is threatened, the current rate of extinction 100 times higher than background rate.
The main cause is habitat loss and fragmentation.
% On one hand it is important to understand how the external environment affect community dynamics
% On the other hand,  we want to know what traits makes species more vulnerable to external environment.


How species responds to habitat destruction is longstanding problem.
In conservation, the species-area relationship informs about how many species will be lost for a given amount of habitat.
A general rule of thumb is that destroying 90\% will kill about 50\% of the community.
The problem is that it does not really tell, which species will go extinct.

One can then look at general traits that make species more vulnerable.
Probably the most important trait is the degree of specialization.
Specialist species are more vulnerable to habitat loss and fragmentation.
Partly because they have narrower ranges, lower abundances, and higher chances that their habitat will be destroyed.

An important concept is the extinction threshold.
It measures the amount of resource required for the persistence of a population.
The extinction threshold depends on many things: the configuration of the landscape, the biology of the species, ...
Hanski et al. developed a convenient measures that takes into account all these things.
In a theoretical work by Tilman, they found that species with lowest colonization rates (due to a trade-off: they are also the best competitor) will
extinct first.

A second critical issue with extinction is that it is not instantaneous.
It takes time for some species to go extinct.
This is called transient time, because of that ghost species that will eventually go extinct.
Hanski found for instance that transient time gets longer as one approaches the extinction threshold.

% There should be work on specialization and extinction threshold here

Currently, there is no work/general prediction that relates these 3 concepts: specialization, extinction threshold, and transient time.
Here we use an individual based model to investigate the relationship among these three quantities.
First, we ask without competition, how level specialization correlates with extinction threshold
Second, we look at how transient time changes with niche width.
Third, we ask how competitive interaction influences their respective extinction threshold.
I will then relate these to data to give a prediction of how species and communities will respond..

\section{Material and methods}
\label{sec:Methods}
We use the same model as in Ramiadantsoa.
The quality/quantity of the landscape can be broken into fragmentation level: size of the patch, quantity of resource within patches, and the type of resource
found in that patch (this is controlling the degree of aggregation of resource types).


Extinction threshold: ameliorate the quality of the landscape (adding resource or reducing fragmentation) until the species persists. (also look at the variance)
The supplementary file will contain figures where dispersal range depends on niche width.
Transient time: run something above the extinction threshold then destroy the habitat (reducing resource production rate or increase mortality rate
and increasing fragmentation).

In the community, we look at the effect of competition, the strength of interaction on the extinction threshold.

Technical details: C++, wrapped as a torus, we look at the total amount of resource not the mean, the relationship looks the same, hopefully!
We assume that things have equilibrate if we duplicate times and nothing changes.

Data:
To illustrate to model, we look at various data, one from Brazil and one from Madagascar.


\section{Results}
\label{sec:Results}
Figure 1 shows that as expected, extinction threshold increases with decreasing niche width.
However the magnitude of changes depends on one key property of the landscape: resource type aggregation.
If resource types are highly aggregated: there is slope is pretty low.

The question here is whether I can estimate that slope just by looking at the fitness function.

In absolute value, there is higher variance for extinction threshold for higher niche width.
But when we look at the normalized standard deviation, things are the same, independent of niche or aggregation.
